\begin{figure}
    \centering
    \includegraphics[width=\columnwidth]{figures/dstat.space.png}
    \caption{Example of a Dstat website showing the traffic received on a testing endpoint. \textbf{TODO: ananymize and better image.}}
    \label{fig:dstat_example}
\end{figure}

\section{Background}

In a \textbf{Distributed Denial of Service (DDoS) attack}, an attacker aims to exhaust resources on a vicitm system, making it unable to respond to legitimate users.
To be able to perform such an attack, the attacker needs to send a lot of network traffic to the victim from different geographical locations around the world.
There exist different ways to send traffic from many places around the world to a single victim.
One of these options is to build a \textbf{botnet}: a network of compromised devices under the control of the attacker.
When launching a DDoS attack, the attacker then instructs all the bots in the botnet to start sending traffic to the victim.

Another option to achieve geographically distributed traffic is to \textbf{spoof} the source of the traffic.
This fakes the source of the data, making it look like it's coming from many places around the world at the victim's side.
While it is becoming more and more difficult to send spoofed traffic on the internet, DDoS attacks actively rely on this mechanism to be effective.

Since the technical requirements to perform a successful DDoS attack are quite complex, markets exist where criminals posessing of these networks can sell DDoS attacks to anyone who wants to perform an attack.
This \textbf{DDoS-for-Hire} market has grown significantly in the last years, and allows anyone to launch a big attack with the click of a button.

Having a market where DDoS attacks are sold implies that there is competition between different service providers.
Users want to know which service provider is capable of providing the strongest attacks for the lowest prices.
However, to show the power of a given DDoS-for-Hire newtork, there needs to be a trused third party reporting the real-world power of a performed attack.

For this reason, DDoS-for-Hire providers have built online platforms where DDoS attacks can be ``tested''. One such platform is shown in Figure~\ref{fig:dstat_example}.
The website presents the user with an IP address and a port, and a graph on the page shows the observed traffic while the attack is happening.
In the DDoS-for-Hire community, these testing endpoints are called \textbf{``dstats''}.
Many of these Dstat websites exist, each presenting a large set of available testing endpoints.
