\section{Market Dynamics}
With a rise in DDoS-for-Hire services \textbf{CITE}, users of these services are provided with a choice between many different providers. Before committing and paying for a service, it is often unclear whether the service will be able to deliver the attack power that is promised. DStat services provide a way for users to test and compare the capabilities of different DDoS-for-Hire services, and are therefore a pivotal component in the ecosystem as they introduce an element of trust to the DDoS-for-Hire market. In this section, we analyze the market dynamics of DDoS Power Proofs on Telegram, focusing on how users interact with DStat bots, the role of competition, and the promotional strategies employed by DDoS-for-Hire providers.


\subsection{Telegram as a marketplace for DDoS-for-Hire services}
\color{red}Check Assessing the aftermath sectie 5.3, ik denk dat we hier ook iets met zon model moeten doen\color{black}.

As shown in Section~\ref{sec:methodology}, Telegram is widely used as a platform for communication and promotion of DDoS-for-Hire services. Many DStat services have dedicated Telegram groups or channels where users can discuss their experiences, share DDoS Power Proofs, and promote their services. The DStat bots in these groups facilitate user interaction with DStat services, allowing users to request attacks and receive real-time feedback on their performance, and creating leaderboards indicating which Telegram user is capable of generating the most traffic on DStat endpoints.

\color{red}CONTINUE HERE\color{black}


\subsection{Race to the Top}
The competitive nature of DDoS-for-Hire services is evident in the way users interact with DStat bots on Telegram. Users often compete to achieve the highest attack volumes on DStat endpoints, with leaderboards showcasing the top performers. This competition drives users to seek out the most powerful DDoS-for-Hire services, as they aim to outperform their peers and gain recognition within the community. Users boast about their attack power, and talk down competitors when they attack the DStat endpoints.

\color{red}CONTINUE HERE\color{black}



\begin{itemize}
    \item Telegram used for most communication
    \item Groups belonging to dstat websites where users can talk freely
    \item Bots to show leaderboards and start dstat counts to end up on those leaderboards
    \item Bot messages forwarded in these channels
    \item Users bound to traffic in dstat bots
    \item Users have links to stressers in their usernames and bio
    \item Constant battle between users to be the best
\end{itemize}