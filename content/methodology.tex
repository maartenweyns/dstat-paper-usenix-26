\section{Methodology}
\label{sec:methodology}

Our methodology consists of two main components: monitoring DStat services to measure their reported uplink capacities and current bandwidth usage, and scraping Telegram groups where DDoS attacks are discussed and advertized to understand the ecosystem of DStat services and their users. To do this, we first identify how DStat services operate, and how they obtain the data they display. We then set up a monitoring infrastructure to scrape DStat services regularly, and collect data over a period of time. Finally, we monitor Telegram groups and channels related to DStat services to gain insights into their operations and user base.

\subsection{DStat monitoring}
DStat websites often boast large uplinks, starting at 1 Gbps and going up to \textbf{TODO}. This infrastructure is expensive to maintain, and requires significant investment in hardware and bandwidth. This raises the question of how these websites can afford such infrastructure, especially when many DStat services are offered for free and rely on donations or ad revenue.

To understand the infrastructure behind DStat websites we characterize all hosts that are listed on DStat services using data from Censys~\textbf{CITE}. We find that all DStat endpoints are exposing either Netdata or Prometheus metrics openly on the internet. Both Netdata and Prometheus are popular open-source monitoring solutions that provide real-time insights into system performance and resource usage. While these should be secured behind firewalls, we find that these hosts allow public access to their metrics endpoints, which can be scraped for data. As these hosts can trivially be scraped, the author of a DStat service can easily collect metrics from these hosts to display on their website, meaning that the DStat services rely on infrastructure that is already publicly exposed rather than their own servers. We find further validation of this hypothesis by observing Telegram conversations among DStat users and operators. In these conversations, DStat ``endpoints'' are often shared, which are simply URLs to Netdata or Prometheus metrics endpoints. This indicates that DStat services are not hosting their own infrastructure, but rather aggregating data from existing publicly available monitoring endpoints. Figure~\textbf{TODO} shows one of these conversations, where a user shares a Netdata endpoint to be added to a DStat service.

The usage of publicly exposed Netdata and Prometheus endpoints for DStat services makes it trivial to set up and maintain a DStat website, as the operator does not need to invest in their own infrastructure, and can use endpoints all over the world. From a scan of the IPv4 address space by Censys \textbf{CITE}, we find \textbf{TODO} hosts exposing Netdata endpoints, and \textbf{TODO} hosts exposing Prometheus endpoints that could be abused for DStat services.

To gain an understanding of the attack traffic that is generated by the users of DStat services, and to gain an understanding of the capabilities of various DDoS-for-Hire services, we scrape 25 DStat services that are publicly listed and have an uplink of at least \textbf{CHECK} 25 Gbps. We update our list of DStat services every per week, as new services appear and old services disappear frequently. We compile the list manually from five DStat websites (listed in Table~\textbf{TODO}), from \textbf{TODO: Date} until \textbf{TODO: Date}. DStat providers scrape the Netdata or Prometheus endpoints every second. We scrape each DStat service every \textbf{TODO: Time}, collecting the reported uplink capacity and the current bandwidth. From these measurements, we can estimate the total attack bandwidth that DDoS-for-Hire services can generate.



\subsection{Telegram scraping}
To understand the ecosystem of DStat services, we monitor Telegram. We identify DStat services by searching for keywords such as "DStat", "DDoS", and "booters". We join public groups and channels related to DStat services, and monitor their activity over a period of \textbf{TODO: Time}. We collect messages, user interactions, and shared links to DStat services. This allows us to gain insights into the popularity of different DStat services, user feedback, and the overall dynamics of the DStat ecosystem on Telegram. We also identify key influencers and operators within these groups who promote or manage DStat services. This qualitative data complements our quantitative measurements from the DStat monitoring, providing a holistic view of the DStat landscape. Table~\textbf{TODO} lists the Telegram groups and channels we monitored during our study.


\subsubsection{DStat Bots and DDoS Power Proofs}
One notable aspect of Telegram groups related to DStat services is the presence of DStat bots. These bots are automated accounts that allow users to interact with DStat services directly from Telegram. Users can request specific endpoints to attack, after which the bot will monitor this specific endpoint to report the observed traffic back in Telegram. These bots are used to test and showcase the power of DDoS-for-Hire services, with attacks being added to daily leaderboards to gamify the experience. Figure~\textbf{TODO} shows an example of a DStat bot interaction in a Telegram group, where a user requests an attack on a specific endpoint and the bot responds with the observed traffic statistics. The results of these bot interactions are often referred to as \textbf{DDoS Power Proofs.}