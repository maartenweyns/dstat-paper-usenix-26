%%%%%%%%%%%%%%%%%%%%%%%%%%%%%%%%%%%%%%%%%%%%%%%%%%%%%%%%%%%%%%%%%%%%%%%%%%%%%%%%
% Template for USENIX papers.
%
% History:
%
% - TEMPLATE for Usenix papers, specifically to meet requirements of
%   USENIX '05. originally a template for producing IEEE-format
%   articles using LaTeX. written by Matthew Ward, CS Department,
%   Worcester Polytechnic Institute. adapted by David Beazley for his
%   excellent SWIG paper in Proceedings, Tcl 96. turned into a
%   smartass generic template by De Clarke, with thanks to both the
%   above pioneers. Use at your own risk. Complaints to /dev/null.
%   Make it two column with no page numbering, default is 10 point.
%
% - Munged by Fred Douglis <douglis@research.att.com> 10/97 to
%   separate the .sty file from the LaTeX source template, so that
%   people can more easily include the .sty file into an existing
%   document. Also changed to more closely follow the style guidelines
%   as represented by the Word sample file.
%
% - Note that since 2010, USENIX does not require endnotes. If you
%   want foot of page notes, don't include the endnotes package in the
%   usepackage command, below.
% - This version uses the latex2e styles, not the very ancient 2.09
%   stuff.
%
% - Updated July 2018: Text block size changed from 6.5" to 7"
%
% - Updated Dec 2018 for ATC'19:
%
%   * Revised text to pass HotCRP's auto-formatting check, with
%     hotcrp.settings.submission_form.body_font_size=10pt, and
%     hotcrp.settings.submission_form.line_height=12pt
%
%   * Switched from \endnote-s to \footnote-s to match Usenix's policy.
%
%   * \section* => \begin{abstract} ... \end{abstract}
%
%   * Make template self-contained in terms of bibtex entires, to allow
%     this file to be compiled. (And changing refs style to 'plain'.)
%
%   * Make template self-contained in terms of figures, to
%     allow this file to be compiled. 
%
%   * Added packages for hyperref, embedding fonts, and improving
%     appearance.
%   
%   * Removed outdated text.
%
%%%%%%%%%%%%%%%%%%%%%%%%%%%%%%%%%%%%%%%%%%%%%%%%%%%%%%%%%%%%%%%%%%%%%%%%%%%%%%%%

\documentclass[letterpaper,twocolumn,10pt]{article}
\usepackage{usenix}

% to be able to draw some self-contained figs
\usepackage{tikz}
\usepackage{amsmath}
\usepackage[many]{tcolorbox}    	% for COLORED BOXES (tikz and xcolor included)
\newtcolorbox{takeawaybox}{
    sharpish corners, % better drop shadow
    boxrule = 0pt,
    toprule = 0.5pt, % top rule weight
    enhanced,
    fuzzy shadow = {0pt}{-2pt}{-0.5pt}{0.5pt}{black!35} % {xshift}{yshift}{offset}{step}{options} 
}

% inlined bib file
\usepackage{filecontents}

%-------------------------------------------------------------------------------
\begin{filecontents}{\jobname.bib}
%-------------------------------------------------------------------------------

\end{filecontents}

%-------------------------------------------------------------------------------
\begin{document}
%-------------------------------------------------------------------------------

%don't want date printed
\date{}

% make title bold and 14 pt font (Latex default is non-bold, 16 pt)
\title{\Large \bf Botnet Boasting: Investigating Market Dynamics of DDoS Power Proofs}

%for single author (just remove % characters)
\author{
% {\rm Maarten Weyns}\\
% Delft University of Technology
% \and
% {\rm Second Name}\\
% Second Institution
} % end author

\maketitle


\begin{abstract}
DDoS-for-Hire services allow anyone to launch Distributed Denial of Service (DDoS) attacks with the click of a button. To showcase their capabilities, these services often rely on third-party platforms that provide ``DDoS Power Proofs'', where users can see real-time graphs of the traffic generated during an attack. These platforms, known as DStat services, aggregate data from publicly exposed monitoring endpoints to display the current bandwidth usage and uplink capacities of various DDoS-for-Hire services. In this paper, we investigate the infrastructure behind DStat services, analyze the attacks performed on their endpoints, and explore the market dynamics of DDoS Power Proofs on Telegram. Our findings reveal that DStat services abuse publicly exposed monitoring endpoints, allowing them to operate without significant investment in infrastructure. We observe a wide range of attack sizes and characteristics, shedding light on the capabilities of DDoS-for-Hire services. Furthermore, our analysis of Telegram groups uncovers the social dynamics and promotional strategies employed by DDoS-for-Hire providers.
\end{abstract}


\section{Introduction}

% With increasing power and complexity, Distributed Denial of Service (DDoS) attacks still remain a valid concern.
% The constant cat-and-mouse game between DDoS mitigation systems and DDoS attack methods make these attacks more effective than ever before.

Distributed Denial of Service (DDoS) attacks continue to pose significant threats to online services, despite advancements in mitigation techniques. The ongoing evolution of attack methods and the corresponding development of defense mechanisms create a dynamic landscape where both attackers and defenders are continually adapting their strategies. This cat-and-mouse game results in increasingly sophisticated DDoS attacks that can bypass traditional mitigation systems. Recent trends show that attackers are increasingly capable of reaching Terrabit-per-second (Tbps) levels of traffic \textbf{TODO: CITE}.

Modern DDoS attacks often leverage large-scale botnets, which are networks of compromised devices that can be orchestrated to flood target systems with overwhelming traffic. It is not trivial to build and maintain such botnets to perform a DDoS attack. However, attackers can rent botnets from DDoS-for-Hire services, which provide easy access to powerful attack capabilities for a fee. These services lower the barrier to entry for launching DDoS attacks, making them accessible to a wider range of malicious actors. As a result, the frequency and scale of DDoS attacks have increased significantly in recent years \textbf{TODO: CITE}.

Around this DDoS-for-Hire ecosystem, a market has emerged where many botnet operators offer their services to perform these DDoS-for-Hire attacks. These operators often compete on price, reliability, and the scale of attacks they can facilitate. From a user perspective, selecting a DDoS-for-Hire service can be challenging due to the lack of transparency, and the unknown firepower of the underlying botnet. Users may struggle to assess the effectiveness of different services, leading to uncertainty about the potential impact of their chosen DDoS attack which they will only discover after the payment has been made.

To address this problem of lack of transparency, the underground economy has seen the emergence of DDoS testing services. These services allow users to identify the strength of a DDoS-for-Hire service before committing to a purchase, and are often shared in Telegram groups. These services typically offer a public dashboard where operators can showcase the extent of their DDoS capabilities by attacking what is essentially a trusted third party. This allows potential customers to evaluate the effectiveness of the DDoS-for-Hire service. These testing services are often referred to as ``Dstat'' panels after the Linux monitoring tool \texttt{dstat} \textbf{TODO: Cite}, which is commonly used to display real-time system resource usage statistics.

In this paper, we present a comprehensive study of the Dstat ecosystem and the Telegram platform backing it. We systematically analyze the features and functionalities of Dstat panels, examining how they operate and where the machines come from that are being attacked. Our research aims to shed light on the inner workings of these services, providing insights into their role within the broader DDoS-for-Hire market. By understanding the dynamics of Dstat panels, we can better comprehend the challenges faced by defenders in mitigating DDoS attacks and the implications for cybersecurity as a whole.

The contributions of this paper are as follows:
\begin{itemize}
    \item We provide a detailed analysis of the Dstat ecosystem, including the features and functionalities of Dstat panels.
    \item We investigate the sources of the machines being attacked in Dstat panels, showing the infrastructure supporting these services.
    \item We explore the role of Telegram in facilitating the Dstat ecosystem, showing the interactions between users and Dstat service providers.
\end{itemize}

\begin{figure}
    \centering
    \includegraphics[width=\columnwidth]{figures/dstat.space.png}
    \caption{Example of a Dstat website showing the traffic received on a testing endpoint. \textbf{TODO: ananymize and better image.}}
    \label{fig:dstat_example}
\end{figure}

\section{Background}

In a \textbf{Distributed Denial of Service (DDoS) attack}, an attacker aims to exhaust resources on a vicitm system, making it unable to respond to legitimate users.
To be able to perform such an attack, the attacker needs to send a lot of network traffic to the victim from different geographical locations around the world.
There exist different ways to send traffic from many places around the world to a single victim.
One of these options is to build a \textbf{botnet}: a network of compromised devices under the control of the attacker.
When launching a DDoS attack, the attacker then instructs all the bots in the botnet to start sending traffic to the victim.

Another option to achieve geographically distributed traffic is to \textbf{spoof} the source of the traffic.
This fakes the source of the data, making it look like it's coming from many places around the world at the victim's side.
While it is becoming more and more difficult to send spoofed traffic on the internet, DDoS attacks actively rely on this mechanism to be effective.

Since the technical requirements to perform a successful DDoS attack are quite complex, markets exist where criminals posessing of these networks can sell DDoS attacks to anyone who wants to perform an attack.
This \textbf{DDoS-for-Hire} market has grown significantly in the last years, and allows anyone to launch a big attack with the click of a button.

Having a market where DDoS attacks are sold implies that there is competition between different service providers.
Users want to know which service provider is capable of providing the strongest attacks for the lowest prices.
However, to show the power of a given DDoS-for-Hire newtork, there needs to be a trused third party reporting the real-world power of a performed attack.

For this reason, DDoS-for-Hire providers have built online platforms where DDoS attacks can be ``tested''. One such platform is shown in Figure~\ref{fig:dstat_example}.
The website presents the user with an IP address and a port, and a graph on the page shows the observed traffic while the attack is happening.
In the DDoS-for-Hire community, these testing endpoints are called \textbf{``dstats''}.
Many of these Dstat websites exist, each presenting a large set of available testing endpoints.

\section{Methodology}
\label{sec:methodology}

Our methodology consists of two main components: monitoring DStat services to measure their reported uplink capacities and current bandwidth usage, and scraping Telegram groups where DDoS attacks are discussed and advertized to understand the ecosystem of DStat services and their users. To do this, we first identify how DStat services operate, and how they obtain the data they display. We then set up a monitoring infrastructure to scrape DStat services regularly, and collect data over a period of time. Finally, we monitor Telegram groups and channels related to DStat services to gain insights into their operations and user base.

\subsection{DStat monitoring}
DStat websites often boast large uplinks, starting at 1 Gbps and going up to \textbf{TODO}. This infrastructure is expensive to maintain, and requires significant investment in hardware and bandwidth. This raises the question of how these websites can afford such infrastructure, especially when many DStat services are offered for free and rely on donations or ad revenue.

To understand the infrastructure behind DStat websites we characterize all hosts that are listed on DStat services using data from Censys~\textbf{CITE}. We find that all DStat endpoints are exposing either Netdata or Prometheus metrics openly on the internet. Both Netdata and Prometheus are popular open-source monitoring solutions that provide real-time insights into system performance and resource usage. While these should be secured behind firewalls, we find that these hosts allow public access to their metrics endpoints, which can be scraped for data. As these hosts can trivially be scraped, the author of a DStat service can easily collect metrics from these hosts to display on their website, meaning that the DStat services rely on infrastructure that is already publicly exposed rather than their own servers. We find further validation of this hypothesis by observing Telegram conversations among DStat users and operators. In these conversations, DStat ``endpoints'' are often shared, which are simply URLs to Netdata or Prometheus metrics endpoints. This indicates that DStat services are not hosting their own infrastructure, but rather aggregating data from existing publicly available monitoring endpoints. Figure~\textbf{TODO} shows one of these conversations, where a user shares a Netdata endpoint to be added to a DStat service.

The usage of publicly exposed Netdata and Prometheus endpoints for DStat services makes it trivial to set up and maintain a DStat website, as the operator does not need to invest in their own infrastructure, and can use endpoints all over the world. From a scan of the IPv4 address space by Censys \textbf{CITE}, we find \textbf{TODO} hosts exposing Netdata endpoints, and \textbf{TODO} hosts exposing Prometheus endpoints that could be abused for DStat services.

To gain an understanding of the attack traffic that is generated by the users of DStat services, and to gain an understanding of the capabilities of various DDoS-for-Hire services, we scrape 25 DStat services that are publicly listed and have an uplink of at least \textbf{CHECK} 25 Gbps. We update our list of DStat services every per week, as new services appear and old services disappear frequently. We compile the list manually from five DStat websites (listed in Table~\textbf{TODO}), from \textbf{TODO: Date} until \textbf{TODO: Date}. DStat providers scrape the Netdata or Prometheus endpoints every second. We scrape each DStat service every \textbf{TODO: Time}, collecting the reported uplink capacity and the current bandwidth. From these measurements, we can estimate the total attack bandwidth that DDoS-for-Hire services can generate.



\subsection{Telegram scraping}
To understand the ecosystem of DStat services, we monitor Telegram. We identify DStat services by searching for keywords such as "DStat", "DDoS", and "booters". We join public groups and channels related to DStat services, and monitor their activity over a period of \textbf{TODO: Time}. We collect messages, user interactions, and shared links to DStat services. This allows us to gain insights into the popularity of different DStat services, user feedback, and the overall dynamics of the DStat ecosystem on Telegram. We also identify key influencers and operators within these groups who promote or manage DStat services. This qualitative data complements our quantitative measurements from the DStat monitoring, providing a holistic view of the DStat landscape. Table~\textbf{TODO} lists the Telegram groups and channels we monitored during our study.


\subsubsection{DStat Bots and DDoS Power Proofs}
One notable aspect of Telegram groups related to DStat services is the presence of DStat bots. These bots are automated accounts that allow users to interact with DStat services directly from Telegram. Users can request specific endpoints to attack, after which the bot will monitor this specific endpoint to report the observed traffic back in Telegram. These bots are used to test and showcase the power of DDoS-for-Hire services, with attacks being added to daily leaderboards to gamify the experience. Figure~\textbf{TODO} shows an example of a DStat bot interaction in a Telegram group, where a user requests an attack on a specific endpoint and the bot responds with the observed traffic statistics. The results of these bot interactions are often referred to as \textbf{DDoS Power Proofs.}
\section{DStat Infrastructure}

\begin{itemize}
    \item Netdata and prometheus hosts for L4
    \item Layer 7 infrastructure is unsure, but difficult to publically scrape
    \item Hosts with large uplinks
    \item We scrape 25 dstat hosts
    \item Update our list once or twice per week
    \item Observe five known dstat websites
\end{itemize}





\section{Attacks on DStat}
\label{sec:attacks}

To understand the capabilities of DDoS-for-Hire services that use DStat services to showcase their power, we analyze the attacks that are performed on the DStat endpoints we monitor. To identify these attacks, we look for spikes in the reported bandwidth on the DStat services. We define an attack as any period where the reported bandwidth \textbf{TODO}. During our monitoring period from \textbf{TODO: Date} until \textbf{TODO: Date}, we observe a total of \textbf{TODO} attacks on the DStat endpoints we monitor. In this section, we analyze these attacks to understand their characteristics and the capabilities of the DDoS-for-Hire services that perform them.


\subsection{Attack Frequency}
To understand the frequency of attacks on the DStat endpoints, we analyze the number of attacks observed per day during our monitoring period. Figure~\textbf{TODO} shows a time series of the number of attacks observed.

\color{red}CONTINUE HERE\color{black}


\subsection{Attack Volume}
\label{sec:attack_volume}
To understand the volume of the attacks performed on the DStat endpoints, we analyze the peak bandwidth observed during each attack. Figure~\textbf{TODO} shows a CDF of the peak bandwidth observed during the attacks per category of DStat endpoints. We split these per endpoint category, as different categories have different uplink capacities, which influences the maximum attack size that can be observed. 

\color{red}CONTINUE HERE\color{black}


\subsection{Attack Characteristics}
To further understand the nature of the attacks performed on the DStat endpoints, we analyze various characteristics of the attacks, including their duration, protocols used, and whether they employ IP spoofing. Figure~\textbf{TODO} shows the distribution of attack durations observed during our monitoring period. We observe that most attacks last for \textbf{TODO}, with a few outliers lasting significantly longer. When a user interacts with a DStat bot to request an attack, the bot typically specifies a duration for the attack, which influences the observed duration. For personal testing attacks however, users can choose any duration they want, leading to a wider distribution of attack durations.

To analyze the protocols used in the attacks, we look for signatures in the reported bandwidth data that indicate whether the attack is using TCP, UDP, or ICMP. Figure~\textbf{TODO} shows the distribution of protocols used in the attacks. We observe that \textbf{TODO} of the attacks use UDP, while \textbf{TODO} use TCP, and \textbf{TODO} use ICMP. This is in line with previous studies on DDoS attacks \textbf{SOME CITES}, which have shown that UDP-based attacks are more common due to their ability to generate high volumes of traffic with minimal effort.

To determine whether the attacks employ IP spoofing, we rely on a separate dataset collected from a network telescope that monitors backscatter traffic as per the method of \textbf{CITE original Caida backscatter paper}. By correlating the timing and IP addresses of the attacks observed on the DStat endpoints with the backscatter data, we can identify whether there is backscatter traffic towards the network telescope. Note that this will provide a strict lower bound on the number of spoofed attacks, as the spoofed attack traffic needs to hit the telescope's monitored IP space to be observed. We rely on a telescope with approximately 65.000 IP addresses, which covers $\approx\frac{1}{2^{16}}$ of the entire IP space. A UDP attack of 1 Gbps with packets of 1440 bytes, which is commonly seen in botnet attacks \textbf{TODO: CITE}, will generate $\approx2^{16}$ packets per second. As each packet has a probability of $\approx\frac{1}{2^{16}}$ to hit the network telescope if the packet source is spoofed randomly over the IPv4 space, we will on average see 1 packet in the network telescope from a 1 Gbps attack. Given the attack durations and sizes shown in the previous paragraphs, it is likely that we therefore do observe any attacks that randomly spoof the source IP address of their packets. From this analysis, we find that only \textbf{TODO: Number and percent} of the attacks show evidence of IP spoofing.

\color{red}CONTINUE HERE\color{black}

\color{red}
\textbf{TODOS} - 
Kunnen we iets zeggen over iets van evolutie hier?

Is er iets interessants te zeggen over de aanvallen zelf? 

Zijn er bepaalde patronen te zien in de aanvallen?
\color{black}


\begin{takeawaybox}
\textbf{Takeaways}
\begin{itemize}
    \item ...
\end{itemize}
\end{takeawaybox}

% \begin{itemize}
%     \item How many attacks per day per ...?
%     \item How big are attacks?
%     \item How long do attacks last?
%     \item What protocols do attacks use (tcp/udp/icmp)
%     \item Do attacks use spoofing (backscatter or not?)
% \end{itemize}
\section{Market Dynamics}

\begin{itemize}
    \item Telegram used for most communication
    \item Groups belonging to dstat websites where users can talk freely
    \item Bots to show leaderboards and start dstat counts to end up on those leaderboards
    \item Bot messages forwarded in these channels
    \item Users bound to traffic in dstat bots
    \item Users have links to stressers in their usernames and bio
    \item Constant battle between users to be the best
\end{itemize}
\input{content/related_work}
\input{content/discussion}
\input{content/conclusion}

%-------------------------------------------------------------------------------
\bibliographystyle{plain}
\bibliography{\jobname}

%%%%%%%%%%%%%%%%%%%%%%%%%%%%%%%%%%%%%%%%%%%%%%%%%%%%%%%%%%%%%%%%%%%%%%%%%%%%%%%%
\end{document}
%%%%%%%%%%%%%%%%%%%%%%%%%%%%%%%%%%%%%%%%%%%%%%%%%%%%%%%%%%%%%%%%%%%%%%%%%%%%%%%%

%%  LocalWords:  endnotes includegraphics fread ptr nobj noindent
%%  LocalWords:  pdflatex acks

